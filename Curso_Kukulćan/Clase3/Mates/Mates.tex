\documentclass{article}
\usepackage{lmodern}
\usepackage[T1]{fontenc}
\usepackage[spanish,activeacute]{babel}
\usepackage{mathtools}
\usepackage[hidelinks]{hyperref}

\title{Ejemplo de \LaTeX{}}
\author{Israel G}
%\date{29 de enero de 2010}
\date{\today}


\begin{document}
\maketitle


Y despu'es de experimentar mucho con diferentes t'ecnicas resulta que la ecuaci'on \ref{sumatoria} es una de las f'ormulas generales. w

\begin{equation}
	\label{sumatoria}
	w = \sum_{i=1}^{n} (x_{i}+y_{i})^{2}$$Esto es una prueba entre formulas de las de arriba y las de abajo$$
	\sqrt{w = \sum_{i=1}^{n} (x_{i}+y_{i})^{2}}
\end{equation}

... y como sabemos que la ecuaci'on \ref{limite} tiene una estructura
\begin{equation*}
	\label{limite}
	\lim_{x \to 0} (x^{2} + 2x + 4) = 4
\end{equation*}

se concluye que...

\begin{equation}
	\label{sumatoria}
	w = \sum_{i=1}^{n} (x_{i}+y_{i})^{2}
\end{equation}

\end{document}